
\chapter{March 9}
\section{判别模型和生成模型}
\begin{itemize}
		\item 判别模型
				\begin{itemize}
						\item 确定性判别:$y=f_{\theta}\left( x \right) $
						\item 随机判别:$p_{\theta}\left( y \mid x \right) $
				\end{itemize}
		\item 生成模型:建立联合概率分布(一般不用)
				\[
						p_{\theta}\left( y \mid x \right) = \frac{p_{\theta}\left( x,y \right)}{p_{\theta}\left( x \right)  }
				.\] 
\end{itemize}

\subsection{生成模型}
\begin{itemize}
		\item 频率派,$\theta$ 是具体的一个点,易于计算
		\item 贝叶斯派,$\theta$ 是一个分布
\end{itemize}


\section{线性回归}
一维的线性回归和二次回归都是线性模型:比如
\[
		f\left( x \right) =\theta_0+\theta_1x+\theta_2 x^2 = \theta^{T} \phi\left( x \right) 
.\] 
$\phi$ 实际上是一种feature engineering。


\noindent \textbf{学习目标} 
\[
		\mathcal{L}\left( \theta \right)  = 
		\frac{1}{n} \sum_{i=1}^{n} \mathcal{L}\left( y_{i},f_{\theta}\left( x_{i} \right)  \right) 
.\] 
损失函数选择:min square loss

\noindent \textbf{学习方法}

梯度下降
\[
		\theta_{new} \leftarrow \theta_{old} - \eta \frac{\partial \mathcal{L}\left( \theta \right) }{\partial \theta} 
.\] 

\section{梯度下降}
\subsection{批量梯度下降}
根据整个批量数据的梯度更新数据
\[
		\frac{\partial J\left( \theta \right) }{\partial \theta} =
		-\frac{1}{N} \sum_{i=1}^{N} \left( y_{i}-f_{\theta}\left( x_{i} \right)  \right) x_{i}
.\] 

\subsection{随机梯度下降}
\href{https://en.wikipedia.org/wiki/Stochastic_gradient_descent}{Stochastic Gradient Descent} 
在数据量较大时比批量梯度下降更为优秀。但是学习中存在\textbf{震荡}或不确定性。 

\noindent \textbf{优化目标}
\[
		J^{(i)} \left( \theta \right) = \frac{1}{2} \left( y_{i}-f_{\theta}\left( x_{i} \right)   \right) ^2 
.\] 
\subsection{小批量梯度下降}
前面两种方式的结合,将训练集分为$K$ 个mini-batch。对每一个小批量更新参数
\[
		J^{(k)}\left( \theta \right)  = \frac{1}{2N_{k}} \sum_{i=1}^{N_k} \left( y_i-f_{\theta}\left( x_{i} \right)  \right) ^2
.\] 
\noindent \textbf{优点}
\begin{itemize}
		\item 结合map-reduce可以比较容易实现并行
		\item 更新速度快,不确定性低
\end{itemize}

\subsection{基本搜索步骤}
随机选择初始化参数,走到局部最优值。

\begin{defi}
		$f:\R^{n}\to \R$ 是凸函数:$\mathrm{dom} f$\footnote{dom 指函数定义域} 是一个凸集,并且
		\[
				f\left( tx_1+\left( 1-t \right) x_2 \right) \le  tf\left( x_1 \right) 
				+ \left( 1-t \right) f\left( x_2 \right) 
		.\] 
\end{defi}
而凸函数是一定有最值的。

\section{从代数角度看线性回归}
\noindent \textbf{目标函数} 
\[
		J\left( \bm{\mu}  \right) = \frac{1}{2} \left( \bm{y} - \bm{X\mu} \right)^{T} 
 \left( \bm{y} - \bm{X\mu} \right)
.\]
而梯度为
\[
		\frac{\partial J(\bm{\mu} )}{\partial \mu} 
		= - \bm{X}^{T} \left( \bm{y}-\bm{X\mu} \right) 
.\] 
甚至可以直接求出最优参数
\[
		\frac{\partial J(\bm{\mu} )}{\partial \mu} = 0 \implies \hat{\mu} = \left( \bm{X^{T}X} \right) ^{-1} \bm{X}^{T} \bm{y}
.\] 
但是当$\bm{X}$ 很大时,这是很难计算的。

\noindent \textbf{当$\bm{X^{T}X}$ 为奇异矩阵 }

其逆矩阵无法计算,解决方法
\begin{itemize}
		\item Regulariazation
		\item $J_1\left( \mu \right)  = J\left( \mu \right) + \frac{\lambda}{2} \|\mu\|^2_{2}  $
\end{itemize}
从而
\[
		\hat{\mu}=  \left( \bm{X^{T}X } + \lambda \bm{I} \right) ^{-1} \bm{X^{T} y}
.\] 
\subsection{泛线性模型}
本质上是做一个替换$ \bm x\to \phi\left( \bm x \right) $,
$\phi\left( x \right) $ 是$\R^{d}\to\R^{h}$
的向量函数。加入了$\phi$ 的线性回归也叫核线性回归。

\subsection{核线性回归的矩阵形式}
使用\href{https://zhuanlan.zhihu.com/p/45223109}{线性代数技巧}得到:
\[
		\hat{y} = \Phi \hat{\theta} = \Phi\Phi^{T} \left( \Phi\Phi^{T} + \lambda I_{n} \right) ^{-1} y 
.\]  
只需关心核矩阵
\[
		K = \Phi\Phi^{T} = \{ k\left( x^{(i)}, x^{(j)} \right) \} 
.\] 

\section{最大似然估计}

带高斯白噪声的线性拟合:
\[
		y = f_{\theta} \left( x \right)  + \ep
.\] 
其中$\ep \sim \mathcal{N} \left( 0, \sigma^2 \right) $.

\noindent \textbf{优化目标}

最大似然(Likelyhood). 
\[
		p\left( y \mid x \right) =\frac{1}{\sqrt{2\pi \sigma^2} } e^{- \frac{\ep^2}{2\sigma^2}   }= \frac{1}{\sqrt{2\pi \sigma^2} } e^{- \frac{y-\theta^{T}x }{2\sigma^2}   }
.\] 
最大化这个似然
\[
		\mathrm{max}_{\theta} \prod_{i=1}^{N} p\left( y_{i}  \mid  x_{i} \right)  
.\] 
等价于最小均方误差的学习(取对数即可证明)

\section{分类指标}
分类器有以下几个指标:
\begin{align*}
		\mathrm{Accuracy}  &= \frac{TP + TN}{TP + FP + TN + FN} \\
		\mathrm{Precision}  &=  \frac{TP}{TP+FP} \\
		\mathrm{Recall}  &=  \frac{TP}{TP+FN} 
.\end{align*} 
为了权衡准确率和召回率,可以调整阀值$h$:
\[
\hat{y} = \begin{cases}
		1, \quad p_{\theta}\left( y=1 \mid x \right) > h \\
		0, \quad \text{else}
\end{cases}
.\] 

为了判别分类器好不好,我们有F1 score
\[
F_1 = \frac{2 \times \mathrm{Precision} \times \mathrm{Recall}  }{ \mathrm{Precision} + \mathrm{Recall}  }
.\] 

\subsection{ROC 曲线}
当$h$ 不断提升时,可以做出一条$\frac{TP}{P} \sim \frac{FP}{N}$曲线(
Receiver operating characteristic).
AUC\footnote{一般处于$\left( 0.5,1 \right) $} (Area under ROC curve) 可以用来衡量准确率
,处于$0.75$以上时可以认为分类模型比较可靠。
